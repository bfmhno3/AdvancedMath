\documentclass[lang=cn,10pt]{elegantbook}

\title{高等数学笔记:基于 \LaTeX{} 的个人知识总结}
\subtitle{Advanced Math Notes: Based on \LaTeX{}}

\author{彭正萧 \& PENG Zhengxiao}
\institute{西北农林科技大学}
\date{始于2023年11月19日}
\version{第七版\ 上册(同济大学数学系\ 编)}
\bioinfo{模板}{\href{https://github.com/ElegantLaTeX/}{ElegantLaTeX}}

\extrainfo{不要以为抹消过去,重新来过,即可发生什么改变。—— 比企谷八幡}

\setcounter{tocdepth}{3}

\logo{logo.pdf}
\cover{cover.jpg}

% 本文档命令
\usepackage{tkz-fct}
\usepackage{array}
\newcommand{\ccr}[1]{\makecell{{\color{#1}\rule{1cm}{1cm}}}}

% 修改标题页的橙色带
% \definecolor{customcolor}{RGB}{32,178,170}
% \colorlet{coverlinecolor}{customcolor}

% 定理类环境:definition(定义)、theorem(定理)、lemma(引理)、corollay(推论)、axiom(公理)、postulate(假设)
% 示例类环境:example, problem, exercise
% 提示类环境:note
% 结论类环境:conclusion, assumption, property, remark, solution

\begin{document}

\maketitle
\frontmatter

\tableofcontents

\mainmatter

\chapter{函数与极限}

\section{映射与函数}

本节主要介绍映射、函数及有关概念,函数的性质与运算等。

\subsection{映射}

\begin{definition}
	设\(X\)、\( Y \)是两个非空集合,如果存在一个法则\( f \),使得对\( X \)中每个元素\( x \),按法则\( f \),在\( Y \)中有唯一确定的元素\( y \)与之对应,那么称\( f \)为从\( X \)到\( Y \)的映射,记作
	\[ f:X \rightarrow Y \]
	其中\( y \)称为元素\( x \)(在映射\( f \)下)的像,并记作\( f(x) \),即
	\[ y = f(x) \]
	而元素\( x \)称为元素\( y \)(在映射\( f \)下)的一个原像;集合\( X \)称为映射\( f \)的定义域,记作\( D_{f} \),即\( D_{f} = X \);\( X \)中所有元素的像所组成的集合称为映射\( f \)的值域,记作\( R_{f} \)或\( f(X) \),即
	\[ R_{f} = f(X) = \{ f(x)\ |\ x \in X\}\].
\end{definition}

\subsection{本地免安装使用}

\textbf{免安装}使用方法如下:从 GitHub 或者 CTAN 下载最新版,严格意义上只需要类文件 \lstinline{elegantbook.cls}。然后将模板文件放在你的工作目录下即可使用。这样使用的好处是,无需安装,简便;缺点是,当模板更新之后,你需要手动替换 \lstinline{cls} 文件。

\subsection{发行版安装与更新}

本模板测试环境为 
\begin{enumerate}
  \item Win10 + \TeX{} Live 2022;
  \item Ubuntu 20.04 + \TeX{} Live 2022;
  \item macOS Monterey + Mac\TeX{} 2022。
\end{enumerate}

\TeX Live/Mac\TeX{} 的安装请参考啸行的\href{https://github.com/OsbertWang/install-latex-guide-zh-cn/releases/}{一份简短的关于安装 \LaTeX{} 安装的介绍}。

安装 \TeX{} Live 之后,安装后建议升级全部宏包,升级方法:使用 cmd 或 terminal 运行 \lstinline{tlmgr update --all},如果 tlmgr 需要更新,请使用 cmd 运行 \lstinline{tlmgr update --self},如果更新过程中出现了中断,请改用 \lstinline{tlmgr update --self --all --reinstall-forcibly-removed} 更新,也即

\begin{lstlisting}
tlmgr update --self 
tlmgr update --all
tlmgr update --self --all --reinstall-forcibly-removed
\end{lstlisting}

更多的内容请参考 \href{https://tex.stackexchange.com/questions/55437/how-do-i-update-my-tex-distribution}{How do I update my \TeX{} distribution?}

\subsection{其他发行版本}

由于宏包版本问题,本模板不支持 C\TeX{} 套装,请务必安装 TeX Live/Mac\TeX{}。更多关于 \TeX{} Live 的安装使用以及 C\TeX{} 与 \TeX{} Live 的兼容、系统路径问题,请参考官方文档以及啸行的\href{https://github.com/OsbertWang/install-latex-guide-zh-cn/releases/}{一份简短的关于安装 \LaTeX{} 安装的介绍}。


\section{关于提交}

出于某些因素的考虑,Elegant\LaTeX{} 项目自 2019 年 5 月 20 日开始,\textbf{不再接受任何非作者预约性质的提交}(pull request)!如果你想改进模板,你可以给我们提交 issues,或者可以在遵循协议(LPPL-1.3c)的情况下,克隆到自己仓库下进行修改。


\chapter{ElegantBook 设置说明}

本模板基于基础的 book 文类,所以 book 的选项对于本模板也是有效的(纸张无效,因为模板有设备选项)。默认编码为 UTF-8,推荐使用 \TeX{} Live 编译。

\section{语言模式}
本模板内含两套基础语言环境 \lstinline{lang=cn}、\lstinline{lang=en}。改变语言环境会改变图表标题的引导词(图,表),文章结构词(比如目录,参考文献等),以及定理环境中的引导词(比如定理,引理等)。不同语言模式的启用如下:


% 附录
\appendix
\chapter{常用的导数公式}

\chapter{常用的三角函数公式}
\begin{enumerate}
	\item 常用三角函数关系
	\begin{enumerate}
		\item 倒数关系:\( \tan \alpha \cdot \cot \alpha = \sin \alpha \cdot \csc \alpha = \cos \alpha \cdot \sec \alpha = 1 \).
		\\
		\item 商数关系:\( \tan \alpha = \dfrac{\sin \alpha}{\cos \alpha} \)、\( \cot \alpha = \dfrac{\cos\alpha}{\sin\alpha} \).
		\\
		\item 平方关系:\( \sin^{2}\alpha + \cos^{2}\alpha = 1 \)、\( 1 + \cot^{2}\alpha = \csc^{2}\alpha \)、\( 1 + \tan^{2}\alpha = \sec^{2}\alpha \).
	\end{enumerate}
	\item 诱导公式(\textcolor{red}{奇变偶不变符号看象限})
	\begin{enumerate}
		\item 
		\( \begin{array}{ll}
			\sin(2k\pi + \alpha) = \sin \alpha & \cos(2k\pi + \alpha) = \cos \alpha \\
			\tan(2k\pi + \alpha) = \tan \alpha & \cot(2k\pi + \alpha) = \cot \alpha \\
		\end{array}
		\)
		\item 
		\(
			\begin{array}{ll}
				\sin(\pi + \alpha) = -\sin \alpha & \cos(\pi + \alpha) = -\cos \alpha \\
				\tan(\pi + \alpha) = \tan \alpha &	\cot(\pi + \alpha) = \cot \alpha \\		
			\end{array}
		\)
	\end{enumerate}
\end{enumerate}
\chapter{基本初等函数的图形}
\begin{itemize}
	\item 幂函数
		\begin{figure}[!htb]
			\centering
			\begin{tikzpicture}[scale=1.5]
				\draw[->,black] (0,0) -- (3,0);
				\draw[->,black] (0,0) -- (0,3);
				
				\tkzInit[xmax=2, ymax=2]
				\tkzHLine{2}
				\tkzVLine{2}
%				\tkzFct[samples=1, domain=0:2]{2}
%				\tkzFct[samples=1, domain=0:1,style=dashed]{1}
				\tkzFct[samples=400, domain=0:5]{x**3}
				\tkzFct[samples=400, domain=0:5]{x**2}
				\tkzFct[samples=400, domain=0:5]{x**(1.5)}
%				\tkzVLine{2}
%				\tkzVLine[style=dashed,domain=0:1]{1}
				\tkzFct[samples=2, domain=0:5]{x}
				\tkzFct[samples=400, domain=0:5]{x**(2./3.)}
				\tkzFct[samples=400, domain=0:5]{x**(1./3.)}
				\tkzFct[samples=400, domain=0:5]{x**(.5)}
				
				\draw[dashed,black] (0,1) -- (1,1);
				\draw[dashed,black] (1,0) -- (1,1);
			\end{tikzpicture}
			\qquad\qquad
			\begin{tikzpicture}[scale=1.5]
				\draw[->,black] (0,0) -- (3,0);
				\draw[->,black] (0,0) -- (0,3);
				
				\tkzInit[xmax=2,ymax=2]
				\tkzFct[samples=400,domain=0.1:2]{x**(-1./3)}
				\tkzFct[samples=400,domain=0.1:2]{x**(-0.5)}
				\tkzFct[samples=400,domain=0.1:2]{x**(-1)}
				\tkzFct[samples=400,domain=0.1:2]{x**(-2)}
				\tkzFct[samples=400,domain=0.5:2]{x**(-3)}
				\tkzVLine{2}
				\tkzHLine{2}
				
				\draw[thick,black,dashed] (0,1) -- (1,1);
				\draw[thick,black,dashed] (1,0) -- (1,1);
			\end{tikzpicture}
			\caption{\( y=x^{\mu} \)}
		\end{figure}
	\item 指数函数
		\begin{figure}[!htb]
			\centering
			\begin{tikzpicture}[scale=1]
				\tkzInit[xmax=3,xmin=-3,ymax=5]
				\tkzFct[samples=400,domain=-2:1.5]{exp(x)}
				\tkzFct[samples=400,domain=-1.5:2]{exp(-x)}
				
				\draw[->,black] (-3,0) -- (3,0);
				\draw[->,black] (0,0) -- (0,5);
				\node[below] (O) at (0,0) {\( O \)};
				\node[below] (x) at (3,0) {\( x \)};
				\node[left] (y) at (0,5) {\( y \)};
				\node[left] (1) at (0,1) {\( 1 \)};
				\node[below] (1.) at (1,0) {\( 1 \)};
				\node[right] (>) at (1,2) {\( a > 1 \)};
				\node[left] (<) at (-1,2) {\( 0<a<1 \)};
			\end{tikzpicture}
			\caption{\( y=a^{x} \)}
		\end{figure}
	\item 对数函数
		\begin{figure}[!htb]
			\centering
			\begin{tikzpicture}[scale=1]
				\draw[->, black] (0,-3) -- (0,3);
				\draw[->, black] (0,0) -- (5,0);
				\node[left] at (0,0) {\( O \)};
				\node[below] at (5,0) {\( x \)};
				\node[left] at (0,3) {\( y \)};
				\node[left] at (0,1) {\( 1 \)};
				\node[below] at (1,0) {\( 1 \)};
				\node[below] at (3,0.9) {\( a>1 \)};
				\node[above] at (3,-0.9) {\( 0<a<1 \)};
				
				\tkzInit[xmax=4,ymax=2,ymin=-2]
				\tkzFct[samples=400,domain=0:4]{log(x)}
				\tkzFct[samples=400,domain=0:4]{-log(x)}
			\end{tikzpicture}
			\caption{\( y=\log_{a} x \)}
		\end{figure}
	\newpage
	\item 三角函数
	\begin{enumerate}
		\item 正弦函数
		\begin{figure}[!htb]
			\centering
			\begin{tikzpicture}[scale=1]
				\draw[->,black] (-2.3*pi,0) -- (2.3*pi,0);
				\draw[->,black] (0,-1.5) -- (0,1.5);
				\draw[black] (0,-1) -- (0.1,-1);
				\draw[black] (0,1) -- (0.1,1);
				\node[below] at (0.15,0) {\( O \)};
				\node[left] at (0,1) {\( 1 \)};
				\node[left] at (0,-1) {\( -1 \)};
				\foreach \myx in {-2*pi,-1.5*pi,-1*pi,-0.5*pi,0.5*pi,1*pi,1.5*pi,2*pi}{%
					\draw (\myx,0) -- (\myx,0.1);
				}
				\node[below] at (2.1*pi,0) {\( 2\pi \)};
				\node[below] at (1.5*pi,0) {\( \frac{3\pi}{2} \)};
				\node[below] at (0.9*pi,0) {\( \pi \)};
				\node[below] at (0.5*pi,0) {\( \frac{\pi}{2} \)};
				
				\node[below] at (-1.9*pi,0) {\( -2\pi \)};
				\node[below] at (-1.5*pi,0) {\( -\frac{3\pi}{2} \)};
				\node[below] at (-1.1*pi,0) {\( -\pi \)};
				\node[below] at (-0.5*pi,0) {\( -\frac{\pi}{2} \)};
				
				\tkzInit[xmin=-2.2*pi,xmax=2.2*pi,ymin=-1.5,ymax=1.5]
				\tkzFct[color=black,thick,domain = -2.2*pi:2.2*pi]{sin(x)}
			\end{tikzpicture}
			\caption{\( y = \sin x \)}
		\end{figure}
		\item 余弦函数
		\begin{figure}[!htb]
			\centering
			\begin{tikzpicture}[scale=1]
				\draw[->,black] (-1.8*pi,0) -- (2.8*pi,0);
				\draw[->,black] (0,-1.5) -- (0,1.5);
				\draw[black] (0,-1) -- (0.1,-1);
				%				\draw[black] (0,1) -- (0.1,1);
				\node[below] at (0.15,0) {\( O \)};
				\node[left] at (0,1) {\( 1 \)};
				\node[left] at (0,-1) {\( -1 \)};
				\foreach \myx in {-1.5*pi,-1*pi,-0.5*pi,0.5*pi,1*pi,1.5*pi,2*pi,2.5*pi}{%
					\draw (\myx,0) -- (\myx,0.1);
				}
				\node[below] at (2.5*pi,0) {\( \frac{5\pi}{2} \)};
				\node[below] at (2.1*pi,0) {\( 2\pi \)};
				\node[below] at (1.55*pi,0) {\( \frac{3\pi}{2} \)};
				\node[below] at (1*pi,0) {\( \pi \)};
				\node[below] at (0.5*pi,0) {\( \frac{\pi}{2} \)};
				
				%				\node[below] at (-2*pi,0) {\( -2\pi \)};
				\node[below] at (-1.6*pi,0) {\( -\frac{3\pi}{2} \)};
				\node[below] at (-1*pi,0) {\( -\pi \)};
				\node[below] at (-0.5*pi,0) {\( -\frac{\pi}{2} \)};
				
				\tkzInit[xmin=-1.8*pi,xmax=2.7*pi,ymin=-1.5,ymax=1.5]
				\tkzFct[color=black,thick,domain = -1.8*pi:2.7*pi]{cos(x)}
			\end{tikzpicture}
			\caption{\( y = \cos x \)}
		\end{figure}
		\item 正切函数
		\begin{figure}[!htb]
			\centering
			\begin{tikzpicture}[scale=0.5]
				\draw[->,black] (0,-5) -- (0,5);
				\draw[->,black] (-1.8*pi,0) -- (2.8*pi,0);
				
				\tkzInit[xmin=-1.5*pi,xmax=2.5*pi,ymin=-5,ymax=5]
				\foreach \i in {-1.5,-0.5,0.5,1.5} {
					\tkzFct[domain=\i*pi+0.01:(\i+1)*pi-0.01]{tan(\x)}
				}
				\foreach \i in {-1.5,-0.5,0.5,1.5,2.5}{%
					\tkzVLine[dashed]{\i*pi}
				}
			\end{tikzpicture}
			\caption{\( y = \tan x \)}
		\end{figure}
		\item 余切函数
		\begin{figure}[!htb]
			\centering
			\begin{tikzpicture}[scale=0.5]
				\draw[->,black] (0,-5) -- (0,5);
				\draw[->,black] (-2.3*pi,0) -- (2.3*pi,0);
				\node[below] at (-0.22,0) {\( O \)};
				
				\tkzInit[xmin=-2*pi,xmax=2*pi,ymin=-5,ymax=5]
				\foreach \xxx in {-2,-1,1,2}{%
					\tkzVLine[dashed]{\xxx*pi}
				}
				\foreach \i in {-2,-1,0,1}{%
					\tkzFct[samples=400,domain=\i*pi+0.01:(\i + 1)*pi-0.01]{1/tan(x)}
				}
			\end{tikzpicture}
			\caption{\( y = \cot x \)}
		\end{figure}
		\newpage
		\item 正割函数
		\begin{figure}[!htb]
			\centering
			\begin{tikzpicture}[scale=0.5]
				\draw[->,black] (-1.8*pi,0) -- (1.8*pi,0);
				\draw[->,black] (0,-5) -- (0,5);
				
				\tkzInit[xmin=-1.5*pi,xmax=1.5*pi,ymin=-5,ymax=5]
				\foreach \i in {-1.5,-0.5,0.5} {
					\tkzFct[domain=\i*pi+0.01:(\i+1)*pi-0.01]{1/cos(x)}
				}
				\foreach \i in {-1.5,-0.5,0.5,1.5}{%
					\tkzVLine[dashed]{\i*pi}
				}
			\end{tikzpicture}
			\caption{\( y = \sec x \)}
		\end{figure}
		\item 余割函数
		\begin{figure}[!htb]
			\centering
			\begin{tikzpicture}[scale=0.5]
				\draw[->,black] (-1.3*pi,0) -- (2.3*pi,0);
				\draw[->,black] (0,-5) -- (0,5);
				
				\tkzInit[xmin=-pi,xmax=2*pi,ymax=5,ymin=-5]
				\foreach \i in {-1,1,2}{%
					\tkzVLine[dashed]{\i*pi}
				}
				\foreach \i in {-1,0,1}{%
					\tkzFct[samples=400,domain=\i*pi+0.01:(\i+1)*pi-0.01]{1/sin(x)}
				}
			\end{tikzpicture}
			\caption{\( y = \csc x \)}
		\end{figure}
	\end{enumerate}
	\item 反三角函数
		\begin{enumerate}
			\item 反正弦函数
			\begin{figure}[!htb]
				\centering
				\begin{tikzpicture}[scale=0.5]
					\draw[->,black] (-1.5,0) -- (1.5,0);
					\draw[->,black] (0,-1.1*pi) -- (0,1.1*pi);
					
					\tkzInit[xmin=-1.5,xmax=1.5,ymin=-pi,ymax=pi]
					\tkzFct[samples=400,domain = -1:1]{asin(x)}
				\end{tikzpicture}
				\caption{\( y=\arcsin x \)}
			\end{figure}
			\newpage
			\item 反余弦函数
			\begin{figure}[!htb]
				\centering
				\begin{tikzpicture}[scale=0.5]
					\draw[->,black] (-1.5,0) -- (1.5,0);
					\draw[->,black] (0,-0.6*pi) -- (0,1.7*pi);
					
					\tkzInit[xmin=-1.5,xmax=1.5,ymin=-pi,ymax=pi]
					\tkzFct[samples=400,domain = -1:1]{acos(x)}
				\end{tikzpicture}
				\caption{\( y = \arccos x \)}
			\end{figure}
			\item 反正切函数
			\begin{figure}[!htb]
				\centering
				\begin{tikzpicture}[scale=0.5]
					\draw[->,black] (-5,0) -- (5,0);
					\draw[->,black] (0,-0.6*pi) -- (0,0.7*pi);
					
					\tkzInit[xmin=-4,xmax=4,ymin=-2,ymax=2]
					\tkzFct[samples=400,domain = -4.5:4.5]{atan(x)}
					\foreach \i in {-0.5,0.5}{%
						\tkzHLine[dashed]{\i*pi}
					}
				\end{tikzpicture}
				\caption{\( y = \arctan x \)}
			\end{figure}
			\item 反余切函数
			\begin{figure}[!htb]
				\centering
				\begin{tikzpicture}[scale=0.5]
					\draw[->,black] (0,-1) -- (0,1.65*pi);
					\draw[->,black] (-5,0) -- (5,0);
					\node[below] at (-0.3,0) {\( O \)};
					\node[below] at (5,0) {\( x \)};
					\node[left] at (0,1.65*pi) {\( y \)};
					
					\tkzInit[xmax=4.8,xmin=-4.8,ymin=0,ymax=pi]
					\tkzFct[samples=400,domain = -4.8:0]{atan(1/x)+pi}
					\tkzFct[samples=400,domain = 0:4.8]{atan(1/x)}
					\tkzHLine[dashed]{pi}
				\end{tikzpicture}
				\caption{\( y = \mathrm{arccot}\ x \)}
			\end{figure}
		\end{enumerate}
\end{itemize}
\chapter{几种常见的曲线}

\end{document}
