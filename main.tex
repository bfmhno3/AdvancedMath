\documentclass[lang=cn,10pt]{elegantbook}

\title{高等数学笔记:基于 \LaTeX{} 的个人知识总结}
\subtitle{Advanced Math Notes: Based on \LaTeX{}}

\author{彭正萧 \& PENG Zhengxiao}
\institute{西北农林科技大学}
\date{始于2023年11月19日}
\version{第七版\ 上册(同济大学数学系\ 编)}
\bioinfo{模板}{\href{https://github.com/ElegantLaTeX/}{ElegantLaTeX}}

\extrainfo{不要以为抹消过去,重新来过,即可发生什么改变。—— 比企谷八幡}

\setcounter{tocdepth}{3}

\logo{logo.pdf}
\cover{cover.jpg}

% 本文档命令
\usepackage{tkz-fct}
\usepackage{ulem}
\usepackage{array}
\newcommand{\ccr}[1]{\makecell{{\color{#1}\rule{1cm}{1cm}}}}
\newcommand{\R}{\mathbb{R}}
\newcommand{\arccot}{\mathrm{arccot}}
\newcommand{\sh}{\mathrm{sh}\ }% 双曲正弦
\newcommand{\ch}{\mathrm{ch}\ }% 双曲余弦
\newcommand{\mth}{\mathrm{th}\ }% 双曲正切

% 修改标题页的橙色带
% \definecolor{customcolor}{RGB}{32,178,170}
% \colorlet{coverlinecolor}{customcolor}

% 定理类环境:definition(定义)、theorem(定理)、lemma(引理)、corollay(推论)、axiom(公理)、postulate(假设)
% 示例类环境:example, problem, exercise
% 提示类环境:note
% 结论类环境:conclusion, assumption, property, remark, solution

\begin{document}

\maketitle
\frontmatter

\tableofcontents

\mainmatter

\chapter{函数与极限}

\section{映射与函数}

本节主要介绍映射、函数及有关概念,函数的性质与运算等。

\subsection{映射}

\begin{definition}
	设\(X\)、\( Y \)是两个非空集合,如果存在一个法则\( f \),使得对\( X \)中每个元素\( x \),按法则\( f \),在\( Y \)中有唯一确定的元素\( y \)与之对应,那么称\( f \)为从\( X \)到\( Y \)的映射,记作
	\[ f:X \rightarrow Y \]
	其中\( y \)称为元素\( x \)(在映射\( f \)下)的像,并记作\( f(x) \),即
	\[ y = f(x) \]
	而元素\( x \)称为元素\( y \)(在映射\( f \)下)的一个原像;集合\( X \)称为映射\( f \)的定义域,记作\( D_{f} \),即\( D_{f} = X \);\( X \)中所有元素的像所组成的集合称为映射\( f \)的值域,记作\( R_{f} \)或\( f(X) \),即
	\[ R_{f} = f(X) = \{ f(x)\ |\ x \in X\}\].
\end{definition}

\begin{note}
	\begin{enumerate}
		\item 构成映射的{\heiti 三要素}:
		\begin{enumerate}
			\item 集合\( X \),即{\heiti 定义域}\( D_{f} = X \).
			\item 集合\( Y \),即{\heiti 值域}的范围\( R_{f} \subset Y \).
			\item {\heiti 对应法则\( f \)},使对每个\( x \in X \),有唯一确定的\( y = f(x) \)与之对应.
		\end{enumerate}
		\item 可以一对多,不能多对一:对每个\( x \in X \),元素\( x \)的像\( y \)是唯一的;而对每个\( y \in R_{f} \),元素\( y \)的原像不一定是唯一的;映射\( f \)的值域\( R_{f} \)是\( Y \)的一个子集,即\( R_{f} \subset Y \),不一定\( R_{f} = Y \).
		\item 映射又称\uwave{算子},根据\( X \)、\( Y \)的不同情形,在不同的数学分支中,映射又有不同的惯用名称,例如:
		\begin{enumerate}
			\item {\heiti 泛函}:从非空集\( X \)到数集\( Y \)的映射又称为\( X \)上的\uwave{泛函}。
			\item {\heiti 变换}:从非空集\( X \)到它自身的映射又称为\( X \)上的\uwave{变换}。
			\item {\heiti 函数}:从实数集\( \R \)(或其子集)\( X \)到实数集\( Y \)的映射通常称为定义在\( X \)上的\uwave{函数}。
		\end{enumerate}
	\end{enumerate}
\end{note}

\begin{definition}[逆映射]
	设\( f \)是\( X \)到\( Y \)的单射,则由定义,对每个\( y \in R_{f} \),有唯一的\( x \in X \),适合\( f(x) y \)。于是,我们可定义一个从\( R_{f} \)到\( X \)的新映射\( g \),即
	\[ g: R_{f} \rightarrow X \]
	对每个\( y \in R_{f} \),规定\( g(y) = x \),这\( x \)满足\( f(x) = y \)。这个映射\( g \)称为\( f \)的逆映射,记作\( f^{-1} \),其定义域\( D_{f^{-1}} = R_{f} \),值域\( R_{f^{-1}} = X \)。
\end{definition}

\begin{note}
	只有单射才存在逆映射。
\end{note}

\begin{definition}[复合映射]
	设有两个映射
	\[ g:X \rightarrow Y_{1}, \qquad\qquad f:Y_{2} \rightarrow Z, \]
	其中\( Y_{1} \subset Y_{2} \),则由映射\( g \)和\( f \)可以定出一个从\( X \)到\( Z \)的对应法则,它将每个\( x \in X \)映成\( f[g(x)] \in Z \)。显然,这个对应法则确定了一个从\( X \)到\( Z \)的映射,这个映射称为映射\( g \)个\( f \)构成的{\heiti 复合映射},记作\( f \circ g \),即
	\[ f \circ g: X \rightarrow Z,\ (f \circ g)(x) = f[g(x)],\ x \in X \]
\end{definition}

\begin{note}
	\begin{enumerate}
		\item \textcolor{red}{\( f \circ g \)有意义并不表示\( g \circ f \)也有意义}。
		\item \heiti 即使\( f \circ g \)与\( g \circ f \)都有意义,复合映射\( f \circ g \)与\( g \circ f \)也未必相同。
	\end{enumerate}
\end{note}

\subsection{函数}
\begin{definition}[函数的概念]
	设数集\( D \subset \R \),则称映射\( f:D \rightarrow \R \),为定义在\( D \)上的\uwave{函数},通常简记为
	\[ y = f(x),\ x \in D \]
	其中\( x \)称为\uwave{自变量},\( y \)称为\uwave{因变量},\( D \)称为\uwave{定义域},记作\( D_{f} \),即\( D_{f} = D \)。
\end{definition}

\begin{note}
	\begin{enumerate}
		\item \( f \)表示自变量\( x \)和因变量\( y \)之间的对应法则,而\( f(x) \)表示自变量\( x \)对应的函数值。
		\item 可以直接用因变量的记号来表示函数,即把函数记作\( y = y(x) \)。
		\item 两个函数相同的{\heiti 充分必要条件}是:两个函数的{\heiti 定义域}相同,{\heiti 对应法则}也相同。
	\end{enumerate}
\end{note}

\begin{definition}[函数的有界性]
	设函数\( f(x) \)的定义域为\( D \),数集\( X \subset D \)。如果存在数\( K_{1} \),使得
	\[ f(x) \leqslant K_{1} \]
	对任一\( x \in X \)都成立,那么称函数\( f(x) \)在\( X \)上有\uwave{上界},而\( K_{1} \)称为函数\( f(X) \)在\( X \)上的一个上界。如果存在数\( K_{2} \),使得
	\[ f(x) \geqslant K_{2} \]
	对任一\( x \in X \)都成立,那么称函数\( f(x) \)在\( X \)上有\uwave{下界},而\( K_{2} \)称为函数\( f(x) \)在\( X \)上的一个下界。如果存在正数\( M \),使得
	\[ |\ f(x)\ | \leqslant M \]
	对任一\( x \in X \)都成立,那么称函数\( f(x) \)在\( X \)上有界。如果这样的\( M \)不存在,就称函数\( f(x) \)在\( X \)上无界;这就是说,如果对于任何正数\( M \),总存在\( x_{1} \in X \),使\( |\ f(x_{1})\ | > M \),那么函数\( f(x) \)在\( X \)上无界。
\end{definition}

\begin{note}
	函数\( f(x) \)在\( X \)上有界的{\heiti 充分必要条件}是它在\( X \)上\uwave{既有上界又有下界}。
\end{note}

\begin{definition}[函数的单调性]
	设函数\( f(x) \)的定义域为\( D \),区间\( I \subset D \)。如果对于区间
\end{definition}

% 附录
\appendix
\renewcommand{\thechapter}{\Roman{chapter}}% 修改附录编号样式为罗马数字
\chapter{常用的麦克劳林(Maclaurin)公式(佩亚洛余项)}
\begin{enumerate}
	\item \[ e^{x} = 1 + x + \dfrac{x^{2}}{2!} + \cdots + \dfrac{x^{n}}{n!} + o(x^{n}) \]
	\item \[ \sin x = x - \dfrac{x^{3}}{3!} + \dfrac{x^{5}}{5!} - \dfrac{x^{7}}{7!} + \cdots + (-1)^{n}\dfrac{x^{2n + 1}}{(2n + 1)!} + o(x^{2n + 2}) \]
	\item \[ \cos x = 1 - \dfrac{x^{2}}{2!} + \dfrac{x^{4}}{4!} - \dfrac{x^{6}}{6!} + \cdots + (-1)^{n}\dfrac{x^{2n}}{(2n)!} + o(x^{2n + 1}) \]
	\item \[ \ln(1 + x) = x - \dfrac{x^{2}}{2} + \dfrac{x^{3}}{3} - \dfrac{x^{4}}{4} + \cdots + (-1)^{n}\dfrac{x^{n + 1}}{n + 1} + o(x^{n + 1}) \]
	\item \[ \dfrac{1}{1 - x} = 1 + x + x^{2} + \cdots +x^{n} + o(x^{n}) \]
	\item \[ (1 + x)^{\alpha} = 1 + \alpha x + \dfrac{\alpha(\alpha - 1)}{2!}x^{2} + \cdots + \dfrac{\alpha(\alpha - 1)\cdots(\alpha - n + 1)}{n!}x^{n} + o(x^{n}) \]
	\item \[ \arcsin x = x + \dfrac{1}{6}x^{3} + o(x^{3}) \]
	\item \[ \arctan x = x - \dfrac{1}{3}x^{3} + o(x^{3}) \]
\end{enumerate}
\chapter{常用的导数公式}
\begin{enumerate}
	\item \( (\ C\ )' = 0 \).
	\item \( (\ x^{\mu}\ )' = \mu x^{\mu - 1} \).
	\item \( (\ \sin x\ )' = \cos x \).
	\item \( (\ \cos x \ )' = -\sin x \).
	\item \( (\ \tan x \ )' = \sec^{2}x \).
	\item \( (\ \cot x \ )' = -\csc^{2}x \).
	\item \( (\ \sec x \ )' = \sec x \cdot \tan x \).
	\item \( (\ \csc x \ )' = -\csc x \cdot \tan x \).
	\item \( (\ a^{x}\ )' = a^{x}\ln a\quad(a > 0, a\neq0) \).
	\item \( (\ e^{x}\ )' = e^{x} \).
	\item \( (\ \log^{a}x\ )' = \dfrac{1}{x\ln a}\quad(a>0,a\neq 1) \).
	\item \( (\ \ln x\ )' = \dfrac{1}{x},\quad (\ \ln |x|\ )' = \dfrac{1}{x} \).
	\item \[ (\ \arcsin x\ )' = \dfrac{1}{1 - x^{2}} \].
	\item \[ (\ \arccos x\ )' = -\dfrac{1}{1 - x^{2}} \].
	\item \[ (\ \arctan x\ )' = \dfrac{1}{1 + x^{2}} \].
	\item \[ (\ \arccot\ x\ )' = -\dfrac{1}{1 +x^{2}} \].
	\item \[ (\ \ln |\sin x| + C\ )' = \cot x \].
	\item \[ (\ -\ln |\cos x|+ C\ )' = \tan x \].
	\item \[ \ln \left(x + \sqrt{a^{2} + x^{2}}\right) \].
	\item \[ (\ \ln |\sec x + \tan x|\ )' = \sec x \].
	\item \[ (\ \ln |\csc x - \cot x|\ )' = \csc x \].
	\item \[ (\ \sqrt{a^{2} + x^{2}} \ )' = \dfrac{x}{\sqrt{a^{2} + x^{2}}} \].
	\item \[ (\ \sqrt{a^{2} - x^{2}} \ )' = \dfrac{x}{\sqrt{a^{2} - x^{2}}} \].
	\item \( (\ \ch x + C \ )' = \sh x \).
	\item \( (\ \sh x + C \ )' = \ch x \).\\
	\begin{remark}
		\[ \text{双曲正弦:} \sh x = \dfrac{e^{2} - e^{2}}{2} \] \\
		\[ \text{双曲余弦:} \ch x = \dfrac{e^{2} + e^{2}}{2} \]
	\end{remark}
\end{enumerate}
\chapter{常用的三角函数公式}
\begin{enumerate}
	\item 常用三角函数关系
	\begin{enumerate}
		\item 倒数关系:\( \tan \alpha \cdot \cot \alpha = \sin \alpha \cdot \csc \alpha = \cos \alpha \cdot \sec \alpha = 1 \).
		\\
		\item 商数关系:\( \tan \alpha = \dfrac{\sin \alpha}{\cos \alpha} \)、\( \cot \alpha = \dfrac{\cos\alpha}{\sin\alpha} \).
		\\
		\item 平方关系:\( \sin^{2}\alpha + \cos^{2}\alpha = 1 \)、\( 1 + \cot^{2}\alpha = \csc^{2}\alpha \)、\( 1 + \tan^{2}\alpha = \sec^{2}\alpha \).
	\end{enumerate}
	\item 诱导公式(\textcolor{red}{奇变偶不变符号看象限})
	\begin{enumerate}
		\item 
		\( \begin{array}{ll}
			\sin(2k\pi \pm \alpha) =\pm \sin \alpha & \cos(2k\pi \pm \alpha) = + \cos \alpha \\
			\tan(2k\pi \pm \alpha) =\pm \tan \alpha & \cot(2k\pi\pm \alpha) = \pm \cot \alpha \\
		\end{array}
		\)
		\item 
		\(
			\begin{array}{ll}
				\sin(\pi \pm \alpha) =  \mp \sin \alpha & \cos(\pi \pm \alpha) = - \cos \alpha \\
				\tan(\pi \pm \alpha) = \pm \tan \alpha & \cot(\pi \pm \alpha) = \pm \cot \alpha \\		
			\end{array}
		\)
		\item
		\(
			\begin{array}{ll}
				\sin(-\alpha) = -\sin\alpha & \cos(-\alpha) = \cos\alpha \\
				\tan(-\alpha) = -\tan\alpha & \cot(-\alpha) = -\cot \alpha \\
			\end{array}
		\)
		\begin{remark}
			三角函数的{\heiti 奇偶性}
		\end{remark}
		\item 
		\(
			\begin{array}{ll}
				\sin(\frac{\pi}{2} \pm \alpha) = + \cos\alpha & \cos(\frac{\pi}{2} \pm \alpha) = \mp \sin \alpha \\
				\tan(\frac{\pi}{2} \pm \alpha) = \mp \cot\alpha & \cot(\frac{\pi}{2} \pm \alpha) = \mp\tan\alpha \\
			\end{array}
		\)
	\end{enumerate}
	\item 二角和差公式
	\begin{enumerate}
		\item \( \cos (\alpha \pm \beta) = \cos\alpha\cdot\cos\beta \mp \sin\alpha\cdot\sin\beta \).
		\item \( \sin(\alpha \pm \beta) = \sin\alpha\cdot\cos\beta \mp \cos\alpha\cdot\sin\beta \).
		\item \( \tan(\alpha \pm \beta) = \displaystyle\dfrac{\tan\alpha \pm \tan\beta}{1 \mp \tan\alpha\cdot\tan\beta} \).
	\end{enumerate}
	\item 积化和差公式
	\begin{enumerate}
		\item \( \sin\alpha\cdot\cos\beta = \frac{1}{2}[\sin(\alpha + \beta) + \sin(\alpha - \beta)] \).
		\item \( \cos\alpha\cdot\sin\beta = \frac{1}{2}[\sin(\alpha + \beta) - \sin(\alpha - \beta)] \).
		\item \( \cos\alpha\cdot\cos\beta = \frac{1}{2}[\cos(\alpha + \beta) + \cos(\alpha - \beta)] \).
		\item \( \sin\alpha\cdot\sin\beta = -\frac{1}{2}[\cos(\alpha + \beta) - \cos(\alpha - \beta)] \).
	\end{enumerate}
	\item 和差化积公式
	\begin{enumerate}
		\item \( \sin\alpha + \sin\beta = 2\sin\dfrac{\alpha + \beta}{2} \cdot \cos\dfrac{\alpha - \beta}{2} \).
		\item \( \sin\alpha - \sin\beta = 2\cos\dfrac{\alpha + \beta}{2} \cdot \sin\dfrac{\alpha - \beta}{2} \).
		\item \( \cos\alpha + \cos\beta = 2\cos\dfrac{\alpha + \beta}{2} \cdot \cos\dfrac{\alpha + \beta}{2} \).
		\item \( \cos\alpha - \cos\beta = -2\sin\dfrac{\alpha + \beta}{2} \cdot \cos\dfrac{\alpha - \beta}{2} \).
	\end{enumerate}
	\item 二倍角公式
	\begin{enumerate}
		\item \( \sin 2\alpha = 2\sin\alpha \cdot \cos \beta \).
		\item \( \cos2\alpha = \cos^{2}\alpha - \sin^{2}\alpha = 2\cos^{2}\alpha - 1 = 1 - \sin^{2}\alpha \).
		\item \( \displaystyle\tan2\alpha = \dfrac{2\tan\alpha}{1 - \tan^{2}\alpha} \).
	\end{enumerate}
	\item 降幂公式
	\begin{enumerate}
		\item \( \cos^{2}\alpha = \frac{1}{2}(1 + \cos2\alpha) \).
		\item \( \sin^{2}\alpha = \frac{1}{2}(1 - \cos2\alpha) \).
	\end{enumerate}
	\item 半角公式
	\begin{enumerate}
		\item \( \tan\frac{\alpha}{2} = \dfrac{\sin\alpha}{1 + \cos\alpha} = \dfrac{1 - \cos\alpha}{\sin\alpha} = \pm \sqrt{\dfrac{1 - \cos\alpha}{1 + \cos\alpha}} \).
		\item \( \displaystyle \cot\frac{\alpha}{2} = \frac{1}{\tan\alpha} \).
	\end{enumerate}
	\item 辅助角公式
	\[  a\sin\alpha \pm b\cos\alpha = \sqrt{a^{2} + b^{2}}\sin(\alpha \pm \varphi),\ \tan\varphi = \frac{b}{a}\].
	\item 万能公式
	\begin{enumerate}
		\item \[ \sin\alpha = \dfrac{2\tan\dfrac{\alpha}{2}}{1 + \tan^{2}\dfrac{\alpha}{2}} \].
		\item \[ \cos\alpha = \dfrac{1 - \tan^{2}\dfrac{\alpha}{2}}{1 + \tan^{2}\dfrac{\alpha}{2}} \].
		\item \[ \tan\alpha= \dfrac{2\tan\dfrac{\alpha}{2}}{1 - \tan^{2}\dfrac{\alpha}{2}}\].
	\end{enumerate}
\end{enumerate}
\chapter{基本初等函数的图形}
\begin{itemize}
	\item 幂函数
		\begin{figure}[!htb]
			\centering
			\begin{tikzpicture}[scale=1.5]
				\draw[->,black] (0,0) -- (2.5,0);
				\node[below] at (2.5,0) {\( x \)};
				\draw[->,black] (0,0) -- (0,2.5);
				\node[left] at (0,2.5) {\( y \)};
				\node[left] at (0,0) {\( O \)};
				\node[left] at (0,1) {\( 1 \)};
				\node[below] at (1,0) {\( 1 \)};
				
				\tkzInit[xmax=2, ymax=2]
				\tkzHLine{2}
				\tkzVLine{2}
%				\tkzFct[samples=1, domain=0:2]{2}
%				\tkzFct[samples=1, domain=0:1,style=dashed]{1}
				\tkzFct[samples=400, domain=0:5]{x**3}
				\tkzFct[samples=400, domain=0:5]{x**2}
				\tkzFct[samples=400, domain=0:5]{x**(1.5)}
%				\tkzVLine{2}
%				\tkzVLine[style=dashed,domain=0:1]{1}
				\tkzFct[samples=2, domain=0:5]{x}
				\tkzFct[samples=400, domain=0:5]{x**(2./3.)}
				\tkzFct[samples=400, domain=0:5]{x**(1./3.)}
				\tkzFct[samples=400, domain=0:5]{x**(.5)}
				
				\draw[dashed,black] (0,1) -- (1,1);
				\draw[dashed,black] (1,0) -- (1,1);
			\end{tikzpicture}
			\qquad\qquad
			\begin{tikzpicture}[scale=1.5]
				\draw[->,black] (0,0) -- (2.5,0);
				\node[below] at (2.5,0) {\( x \)};
				\draw[->,black] (0,0) -- (0,2.5);
				\node[left] at (0,2.5) {\( y \)};
				\draw[dashed,black] (0,1) -- (1,1);
				\draw[dashed,black] (1,0) -- (1,1);
				\node[left] at (0,0) {\( O \)};
				\node[left] at (0,1) {\( 1 \)};
				\node[below] at (1,0) {\( 1 \)};
				
				\tkzInit[xmax=2,ymax=2]
				\tkzFct[samples=400,domain=0.1:2]{x**(-1./3)}
				\tkzFct[samples=400,domain=0.1:2]{x**(-0.5)}
				\tkzFct[samples=400,domain=0.1:2]{x**(-1)}
				\tkzFct[samples=400,domain=0.1:2]{x**(-2)}
				\tkzFct[samples=400,domain=0.5:2]{x**(-3)}
				\tkzVLine{2}
				\tkzHLine{2}
			\end{tikzpicture}
			\caption{\( y=x^{\mu} \)}
		\end{figure}
	\item 指数函数
		\begin{figure}[!htb]
			\centering
			\begin{tikzpicture}[scale=1]
				\tkzInit[xmax=3,xmin=-3,ymax=5]
				\tkzFct[samples=400,domain=-2:1.5]{exp(x)}
				\tkzFct[samples=400,domain=-1.5:2]{exp(-x)}
				
				\draw[->,black] (-3,0) -- (3,0);
				\draw[->,black] (0,0) -- (0,5);
				\node[below] (O) at (0,0) {\( O \)};
				\node[below] (x) at (3,0) {\( x \)};
				\node[left] (y) at (0,5) {\( y \)};
				\node[left] (1) at (0,1) {\( 1 \)};
				\node[below] (1.) at (1,0) {\( 1 \)};
				\node[right] (>) at (1,2) {\( a > 1 \)};
				\node[left] (<) at (-1,2) {\( 0<a<1 \)};
			\end{tikzpicture}
			\caption{\( y=a^{x} \)}
		\end{figure}
	\item 对数函数
		\begin{figure}[!htb]
			\centering
			\begin{tikzpicture}[scale=1]
				\draw[->, black] (0,-3) -- (0,3);
				\draw[->, black] (0,0) -- (5,0);
				\node[left] at (0,0) {\( O \)};
				\node[below] at (5,0) {\( x \)};
				\node[left] at (0,3) {\( y \)};
				\node[left] at (0,1) {\( 1 \)};
				\node[below] at (1,0) {\( 1 \)};
				\node[below] at (3,0.9) {\( a>1 \)};
				\node[above] at (3,-0.9) {\( 0<a<1 \)};
				
				\tkzInit[xmax=4,ymax=2,ymin=-2]
				\tkzFct[samples=400,domain=0:4]{log(x)}
				\tkzFct[samples=400,domain=0:4]{-log(x)}
			\end{tikzpicture}
			\caption{\( y=\log_{a} x \)}
		\end{figure}
	\newpage
	\item 三角函数
	\begin{enumerate}
		\item 正弦函数
		\begin{figure}[!htb]
			\centering
			\begin{tikzpicture}[scale=1]
				\draw[->,black] (-2.3*pi,0) -- (2.3*pi,0);
				\draw[->,black] (0,-1.5) -- (0,1.5);
				\draw[black] (0,-1) -- (0.1,-1);
				\draw[black] (0,1) -- (0.1,1);
				\node[below] at (0.15,0) {\( O \)};
				\node[left] at (0,1) {\( 1 \)};
				\node[left] at (0,-1) {\( -1 \)};
				\foreach \myx in {-2*pi,-1.5*pi,-1*pi,-0.5*pi,0.5*pi,1*pi,1.5*pi,2*pi}{%
					\draw (\myx,0) -- (\myx,0.1);
				}
				\node[below] at (2.1*pi,0) {\( 2\pi \)};
				\node[below] at (1.5*pi,0) {\( \frac{3\pi}{2} \)};
				\node[below] at (0.9*pi,0) {\( \pi \)};
				\node[below] at (0.5*pi,0) {\( \frac{\pi}{2} \)};
				
				\node[below] at (-1.9*pi,0) {\( -2\pi \)};
				\node[below] at (-1.5*pi,0) {\( -\frac{3\pi}{2} \)};
				\node[below] at (-1.1*pi,0) {\( -\pi \)};
				\node[below] at (-0.5*pi,0) {\( -\frac{\pi}{2} \)};
				
				\tkzInit[xmin=-2.2*pi,xmax=2.2*pi,ymin=-1.5,ymax=1.5]
				\tkzFct[color=black,thick,domain = -2.2*pi:2.2*pi]{sin(x)}
			\end{tikzpicture}
			\caption{\( y = \sin x \)}
		\end{figure}
		\item 余弦函数
		\begin{figure}[!htb]
			\centering
			\begin{tikzpicture}[scale=1]
				\draw[->,black] (-1.8*pi,0) -- (2.8*pi,0);
				\draw[->,black] (0,-1.5) -- (0,1.5);
				\draw[black] (0,-1) -- (0.1,-1);
				%				\draw[black] (0,1) -- (0.1,1);
				\node[below] at (0.15,0) {\( O \)};
				\node[left] at (0,1) {\( 1 \)};
				\node[left] at (0,-1) {\( -1 \)};
				\foreach \myx in {-1.5*pi,-1*pi,-0.5*pi,0.5*pi,1*pi,1.5*pi,2*pi,2.5*pi}{%
					\draw (\myx,0) -- (\myx,0.1);
				}
				\node[below] at (2.5*pi,0) {\( \frac{5\pi}{2} \)};
				\node[below] at (2.1*pi,0) {\( 2\pi \)};
				\node[below] at (1.55*pi,0) {\( \frac{3\pi}{2} \)};
				\node[below] at (1*pi,0) {\( \pi \)};
				\node[below] at (0.5*pi,0) {\( \frac{\pi}{2} \)};
				
				%				\node[below] at (-2*pi,0) {\( -2\pi \)};
				\node[below] at (-1.6*pi,0) {\( -\frac{3\pi}{2} \)};
				\node[below] at (-1*pi,0) {\( -\pi \)};
				\node[below] at (-0.5*pi,0) {\( -\frac{\pi}{2} \)};
				
				\tkzInit[xmin=-1.8*pi,xmax=2.7*pi,ymin=-1.5,ymax=1.5]
				\tkzFct[color=black,thick,domain = -1.8*pi:2.7*pi]{cos(x)}
			\end{tikzpicture}
			\caption{\( y = \cos x \)}
		\end{figure}
		\item 正切函数
		\begin{figure}[!htb]
			\centering
			\begin{tikzpicture}[scale=0.5]
				\draw[->,black] (0,-5) -- (0,5);
				\draw[->,black] (-1.8*pi,0) -- (2.8*pi,0);
				
				\tkzInit[xmin=-1.5*pi,xmax=2.5*pi,ymin=-5,ymax=5]
				\foreach \i in {-1.5,-0.5,0.5,1.5} {
					\tkzFct[domain=\i*pi+0.01:(\i+1)*pi-0.01]{tan(\x)}
				}
				\foreach \i in {-1.5,-0.5,0.5,1.5,2.5}{%
					\tkzVLine[dashed]{\i*pi}
				}
			\end{tikzpicture}
			\caption{\( y = \tan x \)}
		\end{figure}
		\item 余切函数
		\begin{figure}[!htb]
			\centering
			\begin{tikzpicture}[scale=0.5]
				\draw[->,black] (0,-5) -- (0,5);
				\draw[->,black] (-2.3*pi,0) -- (2.3*pi,0);
				\node[below] at (-0.22,0) {\( O \)};
				
				\tkzInit[xmin=-2*pi,xmax=2*pi,ymin=-5,ymax=5]
				\foreach \xxx in {-2,-1,1,2}{%
					\tkzVLine[dashed]{\xxx*pi}
				}
				\foreach \i in {-2,-1,0,1}{%
					\tkzFct[samples=400,domain=\i*pi+0.01:(\i + 1)*pi-0.01]{1/tan(x)}
				}
			\end{tikzpicture}
			\caption{\( y = \cot x \)}
		\end{figure}
		\newpage
		\item 正割函数
		\begin{figure}[!htb]
			\centering
			\begin{tikzpicture}[scale=0.5]
				\draw[->,black] (-1.8*pi,0) -- (1.8*pi,0);
				\draw[->,black] (0,-5) -- (0,5);
				
				\tkzInit[xmin=-1.5*pi,xmax=1.5*pi,ymin=-5,ymax=5]
				\foreach \i in {-1.5,-0.5,0.5} {
					\tkzFct[domain=\i*pi+0.01:(\i+1)*pi-0.01]{1/cos(x)}
				}
				\foreach \i in {-1.5,-0.5,0.5,1.5}{%
					\tkzVLine[dashed]{\i*pi}
				}
			\end{tikzpicture}
			\caption{\( y = \sec x \)}
		\end{figure}
		\item 余割函数
		\begin{figure}[!htb]
			\centering
			\begin{tikzpicture}[scale=0.5]
				\draw[->,black] (-1.3*pi,0) -- (2.3*pi,0);
				\draw[->,black] (0,-5) -- (0,5);
				
				\tkzInit[xmin=-pi,xmax=2*pi,ymax=5,ymin=-5]
				\foreach \i in {-1,1,2}{%
					\tkzVLine[dashed]{\i*pi}
				}
				\foreach \i in {-1,0,1}{%
					\tkzFct[samples=400,domain=\i*pi+0.01:(\i+1)*pi-0.01]{1/sin(x)}
				}
			\end{tikzpicture}
			\caption{\( y = \csc x \)}
		\end{figure}
	\end{enumerate}
	\item 反三角函数
		\begin{enumerate}
			\item 反正弦函数
			\begin{figure}[!htb]
				\centering
				\begin{tikzpicture}[scale=0.5]
					\draw[->,black] (-1.5,0) -- (1.5,0);
					\draw[->,black] (0,-1.1*pi) -- (0,1.1*pi);
					
					\tkzInit[xmin=-1.5,xmax=1.5,ymin=-pi,ymax=pi]
					\tkzFct[samples=400,domain = -1:1]{asin(x)}
				\end{tikzpicture}
				\caption{\( y=\arcsin x \)}
			\end{figure}
			\newpage
			\item 反余弦函数
			\begin{figure}[!htb]
				\centering
				\begin{tikzpicture}[scale=0.5]
					\draw[->,black] (-1.5,0) -- (1.5,0);
					\draw[->,black] (0,-0.6*pi) -- (0,1.7*pi);
					
					\tkzInit[xmin=-1.5,xmax=1.5,ymin=-pi,ymax=pi]
					\tkzFct[samples=400,domain = -1:1]{acos(x)}
				\end{tikzpicture}
				\caption{\( y = \arccos x \)}
			\end{figure}
			\item 反正切函数
			\begin{figure}[!htb]
				\centering
				\begin{tikzpicture}[scale=0.5]
					\draw[->,black] (-5,0) -- (5,0);
					\draw[->,black] (0,-0.6*pi) -- (0,0.7*pi);
					
					\tkzInit[xmin=-4,xmax=4,ymin=-2,ymax=2]
					\tkzFct[samples=400,domain = -4.5:4.5]{atan(x)}
					\foreach \i in {-0.5,0.5}{%
						\tkzHLine[dashed]{\i*pi}
					}
				\end{tikzpicture}
				\caption{\( y = \arctan x \)}
			\end{figure}
			\item 反余切函数
			\begin{figure}[!htb]
				\centering
				\begin{tikzpicture}[scale=0.5]
					\draw[->,black] (0,-1) -- (0,1.65*pi);
					\draw[->,black] (-5,0) -- (5,0);
					\node[below] at (-0.3,0) {\( O \)};
					\node[below] at (5,0) {\( x \)};
					\node[left] at (0,1.65*pi) {\( y \)};
					
					\tkzInit[xmax=4.8,xmin=-4.8,ymin=0,ymax=pi]
					\tkzFct[samples=400,domain = -4.8:0]{atan(1/x)+pi}
					\tkzFct[samples=400,domain = 0:4.8]{atan(1/x)}
					\tkzHLine[dashed]{pi}
				\end{tikzpicture}
				\caption{\( y = \mathrm{arccot}\ x \)}
			\end{figure}
		\end{enumerate}
		
				
\end{itemize}
\chapter{几种常见的曲线}

\end{document}
